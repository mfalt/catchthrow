\documentclass{article}

\usepackage[parfill]{parskip}
\usepackage{graphicx}
\usepackage{amsmath}
\usepackage{tikz}
\usepackage[utf8]{inputenc} % For åäö
\usepackage{afterpage}

\title{"Catch and Throw" ball and beam process\\FRTN01 -- Real-Time Systems}
\author{
Mattias Fält -- faldt.mattias@gmail.com
\and
Lucas Jimbergsson -- lucas@jimbergsson.se
\and
Erik Nossborn -- eki.nossborn@gmail.com
\and
Iulia Stoica -- mariajstoica@gmail.com
\and
\and
Supervisor -- Karl-Erik Årz\'{e}n
}

\setcounter{tocdepth}{2}

\begin{document}
\maketitle
\newpage

\tableofcontents
\newpage

\section{Introduction}
\label{sec:introduction}
% Should state the problem to be solved
The goal of this project is to control a "ball and beam" process (depicted in Figure \ref{fig:process}) in a predetermined sequence of tasks:
\begin{itemize}
\item Dispatch a ball from a ball magazine and let the beam catch it.
\item Determine the weight of the ball (comparing with three predefined possible weights).
\item Depending on the weight do either of the following:
\begin{itemize}
\item \textbf{Small ball}: Let it roll into the little cup to the upper left of the beam.
\item \textbf{Medium ball}: Throw it into a basket on the floor.
\item \textbf{Large ball}: Simply drop it on the floor.
\end{itemize}
\end{itemize}
\begin{figure}
\centering
\includegraphics[width=0.7\textwidth]{figures/process_fig.jpg}
\caption{The "ball and beam" process to control.}\label{fig:process}
\end{figure}

\section{Overall approach}
We initially intended on solving the control part of this project by generating trajectories for the ball and beam.
We would then use a time-varying LQG controller to stabilize the system around this predefined trajectory.
We generated the trajectories by iteratively improving the control signal using a gradient based optimization method.
We needed a model of the entire system for this to be possible, and although we did not end up using this method for controlling the system, we decided to include the model for completeness (see appendix \ref{sec:model}).
The main reason for not using this approach was that we knew that it should be possible to achieve satisfactory results using a cascaded PID controller.
We thus chose the safer and possibly easier solution.


\section{Controller structure} \label{sec:controller_structure}
% Should describe the control design aspects of the project.
When controlling the beam angle without any ball on the beam, we are using a simple PID controller.
When controlling the ball position along the beam, we are instead using a cascaded PID controller setup (see figure \ref{fig:cascaded_pid}).
Our implementation also allows for an additional beam angle reference to be supplied as a feed forward signal if desired, but in our final setup we do not make use of this.
\begin{figure}
\centering
\includegraphics[width=\textwidth]{figures/block_diagram.png}
\caption{The cascaded PID setup.}\label{fig:cascaded_pid}
\end{figure}

As mentioned in the introduction, we have a sequence of tasks to be performed, and each of these (catching, weighing and delivering the ball) will be described in further detail in the following sections.

\subsection{Ball catching}\label{sec:ball_catching}
The ball magazine shown to the left of the beam in figure \ref{fig:process} is equipped with a ball dispatcher solenoid, which can be controlled from our program in order to push away a ball from the magazine.
For the ball to roll onto the beam however, the beam has to be aligned with the dispatcher.

Since the sensor measuring the beam angle is not very reliable, we make use of an additional optical sensor detecting when the beam is aligned with the dispatcher.
Starting at some underestimated angle near the dispatcher, the search for the correct beam angle is done by slowly increasing the beam angle until the optical sensor triggers. A small beam angle bias is then corrected for before dispatching a ball.

\subsection{Ball weighing}\label{sec:ball_weighing}
When the ball rolls onto the beam (from the left), we want to determine its weight by measuring how much control signal\footnote{The control signal is essentially corresponding to a motor torque applied on the beam.} is necessary for keeping the ball near a certain weighing position.
We choose the weighing position to be to the right on the beam, and when controlling the ball to this position we are careful to avoid the ball rolling off the beam.
This is achieved by setting the ball position reference signal to a slowly increasing ramp, ending up at the weighing position.

The torque on the beam caused by a ball's gravity force $mg$ is\footnote{We are ignoring that the rotational center of the beam is not exactly at the mass centre of the ball when $x=0$, but actually slightly above.}
\[
\tau_{ball} = mgx
\]
where $x$ is the ball position. Letting $u$ denote the control signal, and assuming\footnote{The mass of the beam actually introduces an additional torque, since the beam is not rotating around its center of mass.
This can however be ignored for small beam angles, which is the case when balancing the ball at a certain position.} that $k_u u = \tau_{ball}$ for some constant $k_u$, $mg/k_u$ can now be derived from $u$ and $x$.
Since $mg/k_u$ will always be the same for a particular ball, this measured quantity can now be used to indicate which ball is currently on the beam.

As the control signal $u$ can be very noisy, we apply lowpass filtering before deciding the ball size.
We are not as picky regarding the ball position however.
As $mg/k_u$ can be measured for any $x\neq0$, we do not need to wait for the controller to keep the ball entirely still at the weighing position before making the ball size decision.
This is implemented by having a high tolerance for the ball position, and reduces the running time of the weighing a lot.

\subsection{Ball delivery}\label{sec:ball_delivery}
Depending on the detected ball size, we now do different things according to the problem formulation in section \ref{sec:introduction}, explained in more detail in the following sections.

\subsubsection{Small ball}\label{sec:small_ball_delivery}
For the small ball to be rolled into the cup to the upper left from the beam, the ball is first stabilized at the weighing position in order to make the following actions more deterministic.
When stabilizing the ball, the tolerance is now much lower as compared to when weighing.

After the ball is stabilized a heuristically pre-tuned angle reference trajectory (see figure \ref{fig:weighandthrowsmallball}) is applied, making the ball roll into the cup.
No ball position feedback is used.

\subsubsection{Medium ball}\label{sec:medium_ball_delivery}
If the ball is determined to have medium size, it is to be thrown to the right, away from the beam.
Before throwing however, the ball is moved to a "throw position" at the left end of the beam.
As for the weighing position, this has to be done carefully (by a suitable choice of reference, see figure \ref{fig:movemediumball}), avoiding the ball rolling off the beam.

When the ball is considered stationary at the throwing position, a pre-tuned angle reference trajectory throws the ball in the same manner as for the small ball described above. See figure \ref{fig:throwmediumball} for the reference appearance.

\subsubsection{Large ball}\label{sec:large_ball_delivery}
If the ball size is determined to be the largest one, a constant beam angle reference is simply applied, making the beam tilt and dropping the ball on the floor.











\section{Program structure}
% Should describe the main program structure, both from a class and and a real-time perspective. If possible illustrate this with some type of figure.
%Vad blir en bra ordning på rubrikerna? Är rubrikerna bra?
%hur ska vi göra med klasser vi inte skrivit?

\subsection{Miscellaneous classes} 	%Beskrivning av klasser som inte passar in under andra rubriker, tex Main och OpCom
%tror även Monitor bör få en snabb beskrivning här, men de flesta metoder i monitorn bör förklaras i beskrivningen av de klasser som faktiskt använder dem.

\subsubsection{Main.java}
%behöver nog inte beskrivas så ingående

\subsection{Regulator classes}

\subsection{Reference generator classes}

\subsection{Checker classes}
Checker classes provide a method to determine if a state is "OK" in some way. We use this to find out when we should continue to the next part of the catch-throw sequence. These classes are found in the \texttt{checker} package.
\subsubsection{StateChecker.java}
\texttt{StateChecker} is the abstract superclass of all the checkers. It has two methods.

The method \texttt{check(double[])} receives our measured values and returns a boolean describing if the state of the system fulfills some condition, depending on the implementation in subclasses.

\texttt{reset()} is used when activating a checker to reset their internal states, so any previous usage of the object won't  affect the current workings. This method will be empty in subclasses that have no internal state.

%\subsubsection{BallOnBeamChecker.java}
%just nu använder vi inte den här, men det finns förändringar vi kan göra för att den ska funka

\subsubsection{ConstBallChecker.java}

This subclass checks if a ball has been around a specified point on the beam for a sufficient number of samples in a row. It provides the method \texttt{setValue(double)} for choosing the wanted stationary point.

\subsubsection{ConstBeamChecker.java}

Exactly the same as \texttt{ConstBallChecker}, except for the angle of the beam.

\subsubsection{LEDChecker.java}

Checks if the beam is in the pickup position - in other words if the LED is on. This is done by actually reading the digital in channel.

\subsection{Threads}

\subsubsection{RegulThread.java}

\subsubsection{SwitchThread.java}



\section{Operator communication}\label{OpCom}
% Should describe the user interface in the project including a short HowTo description on how to start and operate the program.
The program is started by running the \texttt{Main} class which can be found in the \texttt{main} package and the user interface should pop up shortly.

The user interface consists of a main window called "Ball and beam GUI" which contains three realtime graphs where the upper one and the middle one are plotting the reference values (in black) and measurements (in red) for the beam and the ball respectively and the last one is plotting the control signal. The window also contains two panels where the user can change the controller parameters online, and also a set of buttons which can change the running mode of the program:
\begin{itemize}
\item mode OFF: the controllers are off
\item mode BEAM: the program is running control of the beam's angle only and the user can change the reference value manually
\item mode BALL: the program is running control of both the beam's angle and the ball's position and the user can change the reference values manually
\item mode Sequence: the program is running sequence until stopped, i.e catch a ball, weigh the ball, deliver the ball and get ready for another sequence according to Section \ref{sec:controller_structure}.
\end{itemize}

The user interface also contains two smaller windows called "Angle RefGen" and "Position RefGen" where the user can change the reference values of the beam's angle and the ball's position when the program is running in one of the manual modes, i.e BEAM or BALL mode.
\section{Results}
% A section containing the results. In case the project is a control-oriented project this should include plots of measurement signals, reference signal, and control signal. If the project is more of a real-time nature then this section could contain measurement results of different type.
STEGSVAR BOM/BOLLREGULATOR, KOMMENTAR \\
PLOTTAR FÖR SEQUENCE, FÖRKLARA NÄR VAD HÄNDER \\
BESKRIV VILKA SIGNALER PLOTTARNA VISAR \\

\begin{figure}[h]
\centering
\includegraphics[width=0.7\textwidth]{figures/stepresponsebeam-crop.png}
\caption{Step response for the beam angle. AGGRESSIV PGA ATT ORKA BALANSERA ÄVEN TUNGA KULOR}
\label{fig:stepresponsebeam}
\end{figure}

\begin{figure}[h]
\centering
\includegraphics[width=0.7\textwidth]{figures/stepresponseball1-crop.png}
\caption{Step response for the ball position. KOMMENTERA BRUS I STYRSIGNAL}
\label{fig:stepresponseball}
\end{figure}

\begin{figure}[h]
\centering
\includegraphics[width=0.7\textwidth]{figures/topickupposition-crop.png}
\caption{Controlling the beam angle to search for the appropriate pickup position, as described in section \ref{sec:ball_catching}. The process is started at the first line in the graph, and the LED is detected around the second.}
\label{fig:topickupposition}
\end{figure}

\begin{figure}[h]
\centering
\includegraphics[width=0.7\textwidth]{figures/weighandthrowsmallball-crop.png}
\caption{Weighing and throwing the small ball. The ball enters the beam at the first line, and is finished being weighed around the second one. After being stabilized, it is thrown at the third line.}
\label{fig:weighandthrowsmallball}
\end{figure}

\begin{figure}[h]
\centering
\includegraphics[width=0.7\textwidth]{figures/weighmediumball-crop.png}
\caption{Weighing the medium ball, and the beginning of moving it to prepare for a throw. The ball is detected at the first line, and is finished being weighed at the second.}
\label{fig:weighmediumball}
\end{figure}

%kanske inte jätteintressant?
\begin{figure}[h]
\centering
\includegraphics[width=0.7\textwidth]{figures/movemediumball-crop.png}
\caption{Moving the medium ball to throwing position.}
\label{fig:movemediumball}
\end{figure}

\begin{figure}[h]
\centering
\includegraphics[width=0.7\textwidth]{figures/throwmediumball-crop.png}
\caption{The medium ball arriving at the throwing position, and then getting thrown, at the time point specified by the line.}
\label{fig:throwmediumball}
\end{figure}

\begin{figure}[h]
\centering
\includegraphics[width=0.7\textwidth]{figures/weighanddroplargeball-crop.png}
\caption{Weighing and dropping the large ball. The ball enters the beam at the first line, and is finished being weighed at the second, after which it is dropped.}
\label{fig:weighbandthrowlargeball}
\end{figure}





\afterpage{\clearpage}

\section{Conclusions}

\subsection{Controller performance}
As seen in the step response in figure \ref{fig:stepresponsebeam}, the beam control works well but is quite aggressive.
The controller being aggressive is mainly due to heavy integral action, which is a deliberate choice of ours, in order to manage balancing even the largest ball with acceptable performance.

COMMENTS ON STEP RESPONSES, REF TO FIG \\

\subsection{}


PERFORMANCE OF PID \\
COMMENTING ON NUMERICAL TRAJECTORY SEARCH, AND LQG? \\


\afterpage{\clearpage}

\appendix

\section{Process model}\label{sec:model}
\subsection{System Equations}

\begin{figure}[\textwidth]
\centering
\begin{tikzpicture}[scale=1.5]
\usetikzlibrary{patterns,snakes}
\definecolor{Darkgreen}{rgb}{0,0.4,0}
\tikzstyle{brace} = [decorate, decoration={brace, amplitude=5pt}]

\node[inner sep=0] (v0) at (0,0) {};
\node[inner sep=0] (v2) at (2,0.5) {};
\node[inner sep=0] (v1) at (-1,-0.25) {};

\draw[color=red]  (v1) edge (v2);

\node (v3) at (8,-2) {};
\node (v4) at (0,-2) {};
\node (v5) at (8,0) {};

\draw[ball color=blue] (2,0.6) circle (.1);

\draw[dashed,->] (v2) -- (4,1) node[pos=1,above] {$\dot{x}$};
\draw[dashed] (v0) -- (v5) {};
\draw[dashed] (v0) -- (v4) {};
\draw[brace] (v0) -- (v2) node[pos=0.5,above,yshift=6] {$l=x$};
\draw[brace] (v3) -- (v4) node[pos=0.5,below,yshift=-6] {$d_x$};
\draw[brace] (v5) -- (v3)  node[pos=0.5,right,xshift=5] {$d_y$};

 \draw[color=green] plot[smooth] coordinates {(v2) (4,0.6) (6,0) (8,-2)};
 
 
\node at (1.3,0.17) {$\phi$};

\path[clip] (2,0.5) -- (0,0) -- (2,0);
\node[circle,draw=Darkgreen, minimum size=90pt] at (0,0) (circ) {};

\end{tikzpicture}
\caption{Image text HERE}
\label{fig:throw}
\end{figure}


Ball Equation, where $x$ is the distance of the ball along the beam:

\begin{equation}
m(\ddot{x}-x\dot{\phi}^{2})=-mg\sin(\phi)-\frac{2}{5}m\ddot{x}
\end{equation}


Beam Equation:
\begin{equation}
J_{B}\ddot{\phi}=k_{u}u+mgx\cos(\phi)+k_b\phi
\end{equation}


Let $x_{1}=x,\, x_{2}=\dot{x},\, x_{3}=\phi,\, x_{4}=\dot{\phi}$,
we can then write ball equation as 
\begin{eqnarray*}
m(\dot x_{2}-x_{1}x_{4}^{2}) & = & -mg\sin(x_{3})-\frac{2}{5}m\dot{x}_{2}\\
 & \Leftrightarrow\\
\dot{x}_{2} & = & \frac{5}{7}\left(-g\sin(x_{3})+x_{1}x_{4}^{2}\right)
\end{eqnarray*}


and beam equation as
\[
\dot{x}_{4}=\frac{1}{J}\left(k_{u}u+mgx_{1}\cos(x_3)+k_bx_3\right).
\]


We thus have then nonlinear first order system equations

\[
\begin{pmatrix}\dot{x}_{1}\\
\dot{x}_{2}\\
\dot{x}_{3}\\
\dot{x}_{4}
\end{pmatrix}=\begin{pmatrix}x_{2}\\
\frac{5}{7}\left(-g\sin(x_{3})+x_{1}x_{4}^{2}\right)\\
x_{4}\\
\frac{1}{J}mgx_{1}\cos(x_3)+k_bx_3
\end{pmatrix}+\begin{pmatrix}0\\
0\\
0\\
\frac{k_{u}}{J}
\end{pmatrix}u.
\]

\subsection{Linearization}

Introducing the state $x_{5}=m$ and linearizing the system around
steady state $\bar{x}=(\bar{x}_{1},0,0,0,\bar{m})$ gives the pertubation
equations for $\tilde{x}=x-\bar{x}$: 
\begin{eqnarray*}
\dot{\tilde{x}} & \approx & \begin{pmatrix}0 & 1 & 0 & 0 & 0\\
\frac{5}{7}\bar{x}_{4}^{2} & 0 & -\frac{5g}{7}\cos(\bar{x}_{3}) & 2\bar{x}_{1}\bar{x}_{4} & 0\\
0 & 0 & 0 & 1 & 0\\
\bar{m}\frac{g}{J}\cos(\bar{x}_{3}) & 0 & -\bar{m}\frac{g}{J}\bar{x}_{1}\sin(\bar{x}_{3})+k_{b} & 0 & \frac{g}{J}\bar{x}_{1}\cos(\bar{x}_{3})\\
0 & 0 & 0 & 0 & 0
\end{pmatrix}\tilde{x}+\begin{pmatrix}0\\
0\\
0\\
\frac{k_{u}}{J}\\
0
\end{pmatrix}u=\\
 & = & \begin{pmatrix}0 & 1 & 0 & 0 & 0\\
0 & 0 & -\frac{5g}{7} & 0 & 0\\
0 & 0 & 0 & 1 & 0\\
\bar{m}\frac{g}{J} & 0 & k_{b} & 0 & \frac{g}{J}\bar{x}_{1}\\
0 & 0 & 0 & 0 & 0
\end{pmatrix}\tilde{x}+\begin{pmatrix}0\\
0\\
0\\
\frac{k_{u}}{J}\\
0
\end{pmatrix}u
\end{eqnarray*}



\subsection{Throw Equations}

The equations for a free falling body are

\begin{eqnarray*}
h & = & h_{0}+v_{y_{0}}t-\frac{gt^{2}}{2}\\
d & = & d_{0}+v_{x_{0}}t,
\end{eqnarray*}


which results in the terminal time $t_{T}=(d_{T}-d_{0})/v_{x_{0}}$.

We desire 
\begin{eqnarray*}
h_{T} & = & -d_{y}\\
d_{T} & = & d_{x},
\end{eqnarray*}


and know
\begin{eqnarray*}
h_{0} & = & l\sin(\phi)\\
d_{0} & = & l\cos(\phi),
\end{eqnarray*}


which gives
\begin{eqnarray*}
-d_{y} & = & l\sin(\phi)+v_{y_{0}}t-\frac{gt^{2}}{2}\\
d_{x} & = & l\cos(\phi)+v_{x_{0}}t.
\end{eqnarray*}


The velocity of the ball in the horizontal and vertical directions
are
\begin{eqnarray*}
v_{x} & = & \dot{x}\cos(\phi)-x\dot{\phi}\sin(\phi)\\
v_{y} & = & \dot{x}\sin(\phi)-x\dot{\phi}\cos(\phi),
\end{eqnarray*}


and at the release-time, we have $x=l$. These equations gives
\begin{eqnarray*}
-d_{y} & = & l\sin(\phi)+\left(\dot{x}\sin(\phi)-l\dot{\phi}\cos(\phi)\right)t-\frac{gt^{2}}{2}\\
d_{x} & = & l\cos(\phi)+\left(\dot{x}\cos(\phi)-l\dot{\phi}\sin(\phi)\right)t,
\end{eqnarray*}


and from the second equation we arrive at 
\[
t=\frac{\left(d_{x}-l\cos(\phi)\right)}{\dot{x}\cos(\phi)-l\dot{\phi}\sin(\phi)},
\]


and can thus insert that into the first equation:
\begin{gather*}
-d_{y}=l\sin(\phi)+\frac{\left(\dot{x}\sin(\phi)-l\dot{\phi}\cos(\phi)\right)\left(d_{x}-l\cos(\phi)\right)}{\dot{x}\cos(\phi)-l\dot{\phi}\sin(\phi)}-\frac{g}{2}\left(\frac{\left(d_{x}-l\cos(\phi)\right)}{\dot{x}\cos(\phi)-l\dot{\phi}\sin(\phi)}\right)^{2}.
\end{gather*}



\subsection{Assuming $\dot{\phi}=0$}

We can simplify the equation if we assume stationary beam at realease
($\dot{\phi}=0$) 
\begin{eqnarray*}
-d_{y}\dot{x}^{2}\cos^{2}(\phi) & = & \dot{x}^{2}\cos^{2}(\phi)l\sin(\phi)+\dot{x}^{2}\cos(\phi)\sin(\phi)\left(d_{x}-l\cos(\phi)\right)-\frac{g}{2}\left(d_{x}-l\cos(\phi)\right)^{2}
\end{eqnarray*}
simplify:

\[
-d_{y}\dot{x}^{2}\cos^{2}(\phi)=\dot{x}^{2}\cos(\phi)\sin(\phi)d_{x}-\frac{g}{2}\left(d_{x}-l\cos(\phi)\right)^{2}
\]


and we can then finally solve for $\dot{x}$:

\[
\dot{x}=\left(d_{x}-l\cos(\phi)\right)\frac{\sqrt{g}}{\sqrt{2\cos(\phi)\left(d_{y}\cos(\phi)+d_{x}\sin(\phi)\right)}},
\]

\afterpage{\clearpage}

\begin{thebibliography}{9}
\bibitem{NR}
  Thomas Bewley,
  \emph{Numerical Renaissance: simulation, optimization, \& control}.
  Renaissance Press,
  http://numerical-renaissance.com/NR.pdf

\end{thebibliography}

\end{document}



