\documentclass{article}

\usepackage[parfill]{parskip}

\title{Suggested solution}
\author{Name of Author}

\begin{document}
\maketitle

\section{Regulator structure}


\section{UML}


\section{Operator communication}
\subsection{Plotters}
We will have the following plotters:
\begin{itemize}
\item Beam position (reference, measurement and Kalman filter estimate)
\item Ball position (reference, measurement and Kalman filter estimate)
\item Control signal
\end{itemize}

\subsection{Modes}
We intend to implement three different modes; Beam control, Ball control and finally the entire sequence control scheme (catch, weigh, throw).
\subsubsection{Beam control mode}
In this mode only the beam will be controlled by a PID controller. The user should be able to tune the PID parameters as well as altering the reference value online.

\subsubsection{Ball control mode}
As discussed in section \ref{step2}, either LQG or cascaded PID will be used when weighing the ball, and the ball control mode will use the same controller as when weighing the ball.



\subsubsection{Sequence control mode}

\section{Timetable}



\end{document}
