\documentclass{article}
\title{Suggested solution}
\author{Name of Author}

\begin{document}
\maketitle

\section{Regulator structure}\label{regstruc}
We are planning to use different regulator structures for the different parts of the execution. The idea is that we will use a simple PID controller to move the beam to the pick-up position and then a LQG regulator for the other parts.
\subsection{Step 1, Beam control}
We will use a simple PID controller for the position of the beam in this part. A gradually increasing reference will be given for the beam until indication is given that the beam is in the pick-up position. The reference will then stay constant until the next controller is started. The controller for this part is governed by the BeamRegul class.
\subsection{Step 2, Ball balancing}\label{step2}
This part will be regulated by a time-invariant LQG regulator. The model will thus be linearized around the desired position for the weighing. If this proves to be not robust enough (for all weights of the balls), we might introduce the ball weight as a parameter in the model, which can be estimated by the Kalman-filter. We might switch to a cascaded PID controller if this method is too hard to implement or is taking too much time.
\subsection{Step 3, Ball delivery}
We will have the same structure for all controllers in this part but they will have different reference values. The idea for all of them is to first find a nominal solution to the nonlinear equations and then control the system using time-varying LQG regulators. It might be beneficial to have an intermediate regulator that move the system to a state different from the weighing state, before we exectute these controllers, to get a better initial state for the optimization method described below.
\subsubsection{Finding a solution}
We have several ideas on how to solve this part. We can either use JModelical to find an optimal trajectory similar to what is done in Nonlinear Control Lab 3. If this requires too much time, we might also use a numerical optimization approach in MATLAB, as outlined in for example \cite{NR} chapter 21.1.1. Lastly, if these methods fail or provide unsatisfactory solution, we could make a simple ansatz and find an analytical solution.
\subsubsection{Calculating regulator}
The method above would generate a trajectory that satisfies the specific goal for the current ball. We can then, using MATLAB, solve the differential Riccatti equation for the linearized time-varying system using some simple numerical solver. This would generate a feedback and estimation matrix for each timestep which can be saved to disk.
\subsubsection{Executing regulator}
Each regulator can read its respective feedback and estimation values for each time-step at initiation and will then simply apply them in real time.

\section{UML}


\section{Operator communication}
\subsection{Plotters}
We will have the following plotters:
\begin{itemize}
\item Beam position (reference, measurement and Kalman filter estimate)
\item Ball position (reference, measurement and Kalman filter estimate)
\item Control signal
\end{itemize}

\subsection{Modes}
We intend to implement three different modes; Beam control, Ball control and finally the entire sequence control scheme (catch, weigh, throw).
\subsubsection{Beam control mode}
In this mode only the beam will be controlled by a PID controller. The user should be able to tune the PID parameters as well as altering the reference value online.

\subsubsection{Ball control mode}


\subsubsection{Sequence control mode}

\section{Timetable}

\newpage
\begin{thebibliography}{9}
\bibitem{NR}
  Thomas Bewley,
  \emph{Numerical Renaissance: simulation, optimization, \& control}.
  Renaissance Press,
  http://numerical-renaissance.com/NR.pdf

\end{thebibliography}

\end{document}
