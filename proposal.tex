\documentclass{article}
\title{Suggested solution}
\author{Name of Author}

\begin{document}
\maketitle

\section{Regulator structure}


\section{UML}


\section{Operator communication
}

\section{Timetable}
\subsection{Workflow}
\begin{enumerate}
	\item Get beam into position to receive ball
	\item Weigh ball
	\item Throw ball
\end{enumerate}
\subsection{Schedule}
\begin{tabular}[h]{|l|p{10cm}|}
\hline
Week & Things to do \\ \hline
45 (3-9 Nov) & *Sketch a preliminary solution of the problem and create timetable \\
 & *Code for regulation of beam (part 1) and ball \& beam (part 2) will be ready for testing individually on the real process  \\
 & *Have a precise list with all steps required for creating an LQG regulator, ie theoretical model, Riccati equation, estimation of other parameters that are needed (all the theory, basically). Write all these in the LaTeX document to make report writing easier later \\
 & *Structure for the whole program will hopefully be ready (with potential suggestions from Karl-Erik), ie which classes and threads we will have and how they interact \\
\hline
46 (10-16 Nov) & *Hopefully we can now start testing on the real process, test with the regulator for beam only (part 1) and the one for beam \& ball (part 2), as well as estimation of parameters in the model (needed for stage 2) \\
 & *Parallel work: Two people can work on part 1 and two on part 2 \\
 & *GUI for the program \\
\hline
47 (17-23 Nov) & *Continued work on part 1 and 2 \\
 & *Start testing part 3 \\
\hline
48 (24-30 Nov) & *Refine part 3, Kalman filter, improvements of stuff \\
\hline
49 (1-7 Dec) & *Refine part 3 \\
\hline
50 (8-14 Dec) & *12 Dec hand in report, prepare presentation \\
\hline
51 (15-16 Dec) & *Demonstration and presentation 16:th of Decemeber at 15:15 and 17:15 \\
\hline

\end{tabular}


\end{document}
