
\subsection{Controller performance}
As seen in the step response in figure \ref{fig:stepresponsebeam}, the beam control works well but is quite aggressive.
The controller being aggressive is mainly due to heavy integral action.
This is a deliberate choice of ours, in order to manage balancing even the largest ball with acceptable performance.

For the ball, the step response (see figure \ref{fig:stepresponseball}) takes much more time to settle.
These differences in behaviour are to be expected, since the ball balancing problem is much harder.
Ball balancing being harder is partly because the ball dynamics are unstable, and partly because we are actually dealing with one control problem on top of another.
The quite large overshoot in the ball step response could possibly have been reduced, but probably at the price of problems with stationary errors (induced mainly by small "wells" on the beam, in which the ball can get stuck).

The control signal being noisy (as seen in the ball step response in figure \ref{fig:stepresponseball}, for instance), seem to be caused by noisy measurement signals together with derivative action in the PID-controllers.
For both of these the derivative is lowpass-filtered, and we have been experimenting with the level of smoothing. If the derivative is smoothed too much however, the controller gets a slower response to ball/beam accelerations.
The spurious behaviour of the control signal being noisy or not is harder to explain.
One possibility is that the measurements are sometimes more and sometimes less reliable.
For the ball, we have noticed that at som instants it does not get detected, resulting in a slowly decreasing measurement signal.

\subsection{Evaluation of selected approach}
The selected approach of PID controllers together with heuristically tuned references works over our expectations, since we did not even see it as an alternative in the beginning. Its simplicity (not having any explicit need of a process model for example) is also very appealing.

Our initial model-based approach of time-invariant LQG with numerically optimized state trajectories would however possibly have been more beautiful in its generality and.

Worth noting is also that the PID/heuristic approach has a strong coupling (one affecting the other) between changes in PID parameters and changes in references. For the LQG/trajectory approach however, the coupling between the choice of weight matrices and the choice of reference trajectory specifications would not have been as prominent.

