%Vad blir en bra ordning på rubrikerna? Är rubrikerna bra?
%hur ska vi göra med klasser vi inte skrivit?

\subsection{Miscellaneous classes} 	%Beskrivning av klasser som inte passar in under andra rubriker, tex Main och OpCom
%tror även Monitor bör få en snabb beskrivning här, men de flesta metoder i monitorn bör förklaras i beskrivningen av de klasser som faktiskt använder dem.

\subsubsection{Main.java}
%behöver nog inte beskrivas så ingående

\subsection{Regulator classes}

\subsection{Reference generator classes}

\subsection{Checker classes}
Checker classes provide a method to determine if a state is "OK" in some way. We use this to find out when we should continue to the next part of the catch-throw sequence. These classes are found in the \texttt{checker} package.
\subsubsection{StateChecker.java}
\texttt{StateChecker} is the abstract superclass of all the checkers. It has two methods.

The method \texttt{check(double[])} receives our measured values and returns a boolean describing if the state of the system fulfills some condition, depending on the implementation in subclasses.

\texttt{reset()} is used when activating a checker to reset their internal states, so any previous usage of the object won't  affect the current workings. This method will be empty in subclasses that have no internal state.

%\subsubsection{BallOnBeamChecker.java}
%just nu använder vi inte den här, men det finns förändringar vi kan göra för att den ska funka

\subsubsection{ConstBallChecker.java}

This subclass checks if a ball has been around a specified point on the beam for a sufficient number of samples in a row. It provides the method \texttt{setValue(double)} for choosing the wanted stationary point.

\subsubsection{ConstBeamChecker.java}

Exactly the same as \texttt{ConstBallChecker}, except for the angle of the beam.

\subsubsection{LEDChecker.java}

Checks if the beam is in the pickup position - in other words if the LED is on. This is done by actually reading the digital in channel.

\subsection{Threads}

\subsubsection{RegulThread.java}

\subsubsection{SwitchThread.java}
